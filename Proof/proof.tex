\section{Composition Law Proof}
\begin{normalsize}
Given the law, we can translate the left hand side to equal the right.
\begin{lstlisting}[style=code]
pure (.) <*> u <*> v <*> w = u <*> (v <*> w)
\end{lstlisting}
Taking only the left hand side, and applying the definition of pure for Maybe:
\begin{lstlisting}[style=code]
Pure x = Just x
\end{lstlisting}
The left hand side is left as:
\begin{lstlisting}[style=code]
(Just (.)) <*> u <*> v <*> w 
\end{lstlisting}
We know that for the Maybe type, $<*>$ is defined as:
\begin{lstlisting}[style=code] 	12`	
instance Applicative Maybe where  
    pure = Just  
    Nothing <*> _ = Nothing  
    (Just f) <*> something = fmap f something  
\end{lstlisting}
Hence if either argument for $<*>$ is nothing, the entire expression reduces to Nothing, so we can ignore these and presume u, v and w are something. Using this, we can then apply $<*>$ (being careful to keep terms left-applicative) to the first two terms.
\begin{lstlisting}[style=code]
(Just (.) f) <*> (Just g) <*> (Just x)
\end{lstlisting}
Repeating this:
\begin{lstlisting}[style=code]
(Just (.) f g) <*> (Just x)
\end{lstlisting}
And then applying $<*>$ a final time:
\begin{lstlisting}[style=code]
(Just (f.g) x)
\end{lstlisting}
Which expands to:
\begin{lstlisting}[style=code]
(Just f (g x))
\end{lstlisting}

From here, we can apply:
\begin{lstlisting}[style=code]
pure f <*> pure x = pure (f x)
\end{lstlisting}
We apply the homomorphism law once, and then again to the second generated term. 
\begin{lstlisting}[style=code]
(Just f) <*> Just (g x) 

(Just f) <*> ( (Just g) <*> (Just x) )
\end{lstlisting}

Finally, we can remove the earlier definitions of u, v, w to give the right side of our original equation.
\begin{lstlisting}[style=code]
u <*> (v <*> w)
\end{lstlisting}
\end{normalsize}